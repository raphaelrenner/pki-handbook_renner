\chapter{Certificate Revocation}

\section{CRL – Certificate Revocation Lists}
\subsection*{Funktionsweise von CRL}

Eine \textbf{CRL} (Certificate Revocation List) ist eine Liste, die von einer Zertifizierungsstelle (CA) veröffentlicht wird und alle Zertifikate enthält, die vor ihrem regulären Ablaufdatum widerrufen wurden. Eine CRL dient dazu festzustellen, ob ein Zertifikat noch gültig ist oder ob es aufgrund eines Widerrufs als ungültig betrachtet werden muss. CRLs werden besonders in Systemen eingesetzt, die keine Echtzeitüberprüfung wie OCSP benötigen oder unterstützen.

\subsection*{Soll-Funktionsweise von CRL}

\begin{itemize}
    \item \textbf{Zertifikatsanfrage:} Ein Client (z.\,B. ein Webbrowser) möchte die Gültigkeit eines Zertifikats prüfen. Dazu lädt er eine aktuelle CRL von der Zertifizierungsstelle herunter, statt eine Echtzeitabfrage wie bei OCSP durchzuführen.
    \item \textbf{Abgleich mit der CRL:} Die CRL enthält alle widerrufenen Zertifikate samt Seriennummern. Der Client prüft, ob die Seriennummer des zu überprüfenden Zertifikats in der CRL enthalten ist. Ist dies der Fall, gilt das Zertifikat als widerrufen und damit ungültig.
    \item \textbf{Verwendung der CRL:} Der Client lädt die CRL regelmäßig neu herunter, um stets den aktuellen Widerrufsstatus zu kennen. Die CRL wird in festen Intervallen durch die CA aktualisiert und enthält alle Zertifikate, die in diesem Zeitraum widerrufen wurden.
\end{itemize}

Zusammenfassend lässt sich sagen, dass CRLs eine etablierte Methode zur Überprüfung des Widerrufsstatus von Zertifikaten darstellen. Allerdings kann der Download der gesamten Liste bei großen PKIs ineffizient sein. Dennoch bleibt die CRL ein weit verbreiteter und zuverlässiger Mechanismus, insbesondere in Umgebungen ohne Echtzeitprüfung.


\subsection{Erstellung einer CRL unter Verwendung von XCA}
\subsubsection*{Ausgangssituation und Zielsetzung}
In diesem Labor wird eine Certificate Revocation List (CRL) erstellt und über einen Webserver (Nginx) bereitgestellt. Die vorhandene CA-Struktur wurde bereits zuvor in XCA aufgebaut. Ziel ist es, ein Serverzertifikat mit einem CRL Distribution Point (CDP) zu versehen, dieses anschließend zu widerrufen und die zugehörige CRL über Nginx zugreifbar zu machen. Am Ende soll der Browser beim Inspizieren des Zertifikats korrekt auf den veröffentlichten CRL-Pfad verweisen.

\subsubsection*{Anpassung des Serverzertifikats in XCA}

Öffnen der bestehenden CA-Struktur in XCA.

\begin{figure}[H]
    \centering
    \includegraphics[width=0.8\textwidth]{xcacrl1.png}
    \caption*{XCA Übersicht}
\end{figure}

Auswahl des entsprechenden Serverzertifikats, das angepasst werden soll.

Im Reiter \texttt{Extensions} bzw. \texttt{Extensions → CRL Distribution Points} den Distribution Point eintragen:

\begin{verbatim}
http://192.168.1.1/crl/cacrl.pem
\end{verbatim}

\begin{figure}[H]
    \centering
    \includegraphics[width=0.8\textwidth]{xcacrl2.png}
    \caption*{CRL Distribution Point}
\end{figure}

\subsubsection*{Widerruf des Serverzertifikats}

Das soeben angepasste Serverzertifikat wird in XCA ausgewählt.

Über \texttt{Revocation / Revoke} wird das Zertifikat widerrufen.

\begin{figure}[H]
    \centering
    \includegraphics[width=0.8\textwidth]{xcacrl3.png}
    \caption*{Widerruf in XCA}
\end{figure}

Dadurch wird klar, dass dieses Zertifikat nicht mehr gültig ist und künftig in der CRL eingetragen werden muss.

\subsubsection*{Erstellen der CRL des Intermediate-Zertifikats}

Da das widerrufene Serverzertifikat vom Intermediate-Zertifikat signiert wurde, wird die CRL auf Ebene dieser Intermediate-CA erzeugt:

\begin{itemize}
    \item Auswahl der Intermediate-CA in XCA.
    \item Im Bereich \texttt{CRLs} die Option \texttt{New CRL} wählen.
    \item CRL-Parameter wie Gültigkeitsdauer übernehmen oder anpassen.
    \item CRL erzeugen und als Datei exportieren (PEM-Format).
\end{itemize}

Ausgabeformat:

\begin{verbatim}
cacrl.pem
\end{verbatim}

\begin{figure}[H]
    \centering
    \includegraphics[width=0.8\textwidth]{xcacrl4.png}
    \caption*{CRL Erstellung}
\end{figure}


\subsection{Erstellung einer CRL unter Verwendung von OpenSSL}
\input{chapters/06-Revokation/CRL_XCA_nginx/crlossl}

\subsection{Veröffentlichung einer CRL unter Verwendung von Nginx}
\subsubsection*{Bereitstellen der CRL über Nginx}

\paragraph*{Ablage der CRL auf dem Webserver}
Auf der Ubuntu-VM den vorgesehenen Ordner unter \texttt{/var/www/html/} anlegen (falls nicht vorhanden):

\begin{verbatim}
sudo mkdir -p /var/www/html/crl
\end{verbatim}

Die zuvor exportierte Datei \texttt{cacrl.pem} nach diesem Verzeichnis kopieren:

\begin{verbatim}
sudo cp /pfad/zur/cacrl.pem /var/www/html/crl/
\end{verbatim}

\begin{figure}[h!]
    \centering
    \caption*{Dateistruktur auf der VM}
\end{figure}

\paragraph*{Nginx neu starten}
Nach dem Platzieren der Datei muss Nginx neu geladen werden, damit die CRL korrekt ausgeliefert wird:

\begin{verbatim}
sudo systemctl restart nginx
\end{verbatim}

\begin{figure}[h!]
    \centering
    \caption*{Bestätigung des Neustarts}
\end{figure}

\subsubsection*{Überprüfung im Browser}

\begin{itemize}
    \item Zugriff auf den Webdienst mit dem betroffenen Serverzertifikat herstellen.
    \item Im Browser das Zertifikat inspizieren (Zertifikatsdetails öffnen).
    \item Unter \texttt{CRL Distribution Points} sollte nun der gesetzte Link erscheinen:
\end{itemize}

\begin{verbatim}
http://192.168.1.1/crl/cacrl.pem
\end{verbatim}

Test: Der Link lässt sich im Browser öffnen – die CRL kann theoretisch heruntergeladen werden.

\subsubsection*{Ergebnis}

Durch die korrekte Konfiguration des CRL Distribution Points, das Widerrufen des Serverzertifikats und das Erstellen sowie Bereitstellen der CRL über Nginx kann der Browser die CRL erfolgreich finden und anzeigen. Damit ist der Widerrufsmechanismus funktional implementiert.

\subsection{Veröffentlichung einer CRL unter Verwendung von Apache}
\input{chapters/06-Revokation/CRL_XCA_nginx/CRLapache}



\subsection{Verhalten moderner Browser bei CRLs}

\subsubsection*{Einordnung}

Nachdem im vorherigen Abschnitt gezeigt wurde, wie eine CRL mit XCA erzeugt und über einen Webserver wie Nginx bereitgestellt wird, stellt sich in der Praxis die Frage, welche Clients diese Informationen tatsächlich auswerten. Besonders relevant ist dabei das Verhalten moderner Webbrowser, da diese maßgeblich bestimmen, ob ein Zertifikatswiderruf im Alltag zuverlässig erkannt wird.

In realen Systemen zeigt sich jedoch, dass viele Browser die im Zertifikat hinterlegten Widerrufsinformationen – wie CRL Distribution Points oder OCSP-Server – entweder gar nicht oder nur eingeschränkt berücksichtigen. Dies gilt insbesondere in Umgebungen mit eigenen, internen Zertifizierungsstellen, wie in unserem Laboraufbau. Selbst wenn eine CRL technisch korrekt erstellt und öffentlich bereitgestellt wird, heißt das nicht, dass alle Browser diese Informationen automatisch abrufen oder zur Validierung heranziehen.

\subsubsection{Mozilla Firefox}
\input{chapters/06-Revokation/CRL_im_Browser/Firefox.tex}

\subsubsection{Google Chrome}
\input{chapters/06-Revokation/CRL_im_Browser/Chrome.tex}

\subsubsection{Microsoft Edge}
\input{chapters/06-Revokation/CRL_im_Browser/Edge.tex}

\subsubsection{Apple Safari}
\input{chapters/06-Revokation/CRL_im_Browser/Safari.tex}

\subsubsection*{Allgemeine Herausforderungen und Problematik bei Zertifikatswiderrufen}
\input{chapters/06-Revokation/CRL-Problematik/crlproblem.tex}


% ==========================================================
%           **NEUER HAUPTPUNKT: OCSP**
% ==========================================================
\newpage

\section{OCSP – Online Certificate Status Protocol}
\subsection*{Funktionsweise von OCSP}

Ein \textbf{OCSP} (Online Certificate Status Protocol) ist ein Netzwerkprotokoll, welches verwendet wird, um den Status von digitalen Zertifikaten in Echtzeit zu überprüfen. Es dient hauptsächlich der Überprüfung, ob ein Zertifikat von einer Zertifizierungsstelle (CA) als gültig, widerrufen oder ungültig markiert wurde, ohne dass der gesamte Zertifikatstatus heruntergeladen werden muss. OCSP wird oft als eine Alternative zu CRLs (Certificate Revocation Lists) verwendet.

\subsection*{Soll-Funktionsweise von OCSP}

\begin{itemize}
    \item \textbf{Zertifikatsanfrage:} Ein Client (z.\,B. ein Webbrowser) möchte die Gültigkeit eines Zertifikats prüfen. Anstatt eine vollständige Liste von widerrufenen Zertifikaten (CRL) herunterzuladen, sendet der Client eine Anfrage an einen OCSP-Responder (eine spezielle Stelle, die den Status des Zertifikats prüft).
    \item \textbf{Anfrage an den OCSP-Responder:} Die Anfrage enthält Informationen über das Zertifikat, das überprüft werden soll, einschließlich der Seriennummer des Zertifikats. Diese Anfrage wird an den OCSP-Responder geschickt.
    \item \textbf{Antwort des OCSP-Responders:} Der OCSP-Responder antwortet mit einer Statusmeldung, die anzeigt, ob das Zertifikat gültig, widerrufen oder unbekannt ist. Die Antwort enthält auch einen Zeitstempel, um sicherzustellen, dass die Antwort aktuell ist.
    \item \textbf{Überprüfung der Antwort:} Der Client überprüft die Antwort des OCSP-Responders. Wenn der Status des Zertifikats als gültig bestätigt wird, kann der Client dem Zertifikat vertrauen. Bei einem widerrufenen oder unbekannten Status wird der Client das Zertifikat ablehnen.
\end{itemize}

Zusammenfassend lässt sich sagen, dass OCSP ein effizientes und praktisches Protokoll ist, um die Gültigkeit von Zertifikaten in Echtzeit zu überprüfen, und eine gute Alternative zu CRLs darstellt, wenn es funktioniert.

\subsection{Aufsetzen eines OCSP-Responders}


Für diesen OCSP-Responder wird ein Modell der öffentlich verfügbaren Software \textbf{XiPKI} verwendet, zu finden unter:
\url{https://github.com/xipki/xipki}

\subsubsection*{Download von XiPKI}

Unter den Releases kann das Setup für die zum Zeitpunkt dieser Anleitung aktuellste Version (6.5.3) heruntergeladen werden.
Hinweis: Für Tomcat wird mindestens Java 11 benötigt.

\url{https://github.com/xipki/xipki/releases/tag/v6.5.3}

Das Archiv enthält eine Datei \texttt{restore.sh}, die mit einem Zielverzeichnis ausgeführt werden muss. In diesem Zielverzeichnis landen anschließend alle Module von XiPKI.

\begin{verbatim}
chmod +x restore.sh
./restore.sh ../xipki/
\end{verbatim}

Nach dem Wechsel in dieses Zielverzeichnis ergibt sich folgende Ordnerstruktur:

\begin{verbatim}
-rw-rw-r-- 1 buchi buchi  24K Jan  1 2024 CHANGELOG.md
-rw-rw-r-- 1 buchi buchi 6.8K Jan  1 2024 INSTALL.md
drwxrwxr-x 2 buchi buchi 4.0K Nov 18 22:31 license
-rwxr-xr-x 1 buchi buchi 3.1K Jan  1 2024 prepare.sh
-rw-rw-r-- 1 buchi buchi 7.5K Jan  1 2024 README.md
drwxrwxr-x 3 buchi buchi 4.0K Nov 18 22:31 setup
drwxrwxr-x 5 buchi buchi 4.0K Nov 18 22:31 xipki-ca
drwxrwxr-x 9 buchi buchi 4.0K Nov 18 22:31 xipki-cli
drwxrwxr-x 5 buchi buchi 4.0K Nov 18 22:31 xipki-gateway
drwxrwxr-x 5 buchi buchi 4.0K Nov 18 22:31 xipki-hsmproxy
drwxrwxr-x 9 buchi buchi 4.0K Nov 18 22:31 xipki-mgmt-cli
drwxrwxr-x 5 buchi buchi 4.0K Nov 18 22:31 xipki-ocsp
\end{verbatim}

Für die Installation wird der Ordner \texttt{xipki-ocsp} benötigt.

\subsubsection*{Installation des OCSP-Responders}

\subsubsection*{Installation von Tomcat}

Eine Installationsanleitung findet sich hier:
\url{https://www.digitalocean.com/community/tutorials/how-to-install-apache-tomcat-10-on-ubuntu-20-04}

Für den OCSP-Server wird Tomcat 10 als Webserver benötigt. Es wird empfohlen, Tomcat unter einem eigenen Benutzer auszuführen:

\begin{verbatim}
sudo useradd -m -d /opt/tomcat -U -s /bin/false tomcat
\end{verbatim}

Anschließend wird die neueste Tomcat-Version heruntergeladen und nach \texttt{/opt/tomcat} verschoben:

\begin{verbatim}
wget https://dlcdn.apache.org/tomcat/tomcat-10/v10.1.49/bin/apache-tomcat-10.1.49.tar.gz
tar -xvzf apache-tomcat-10.1.49.tar.gz

sudo mkdir -p /opt/tomcat
sudo mv apache-tomcat-10.1.49 /opt/tomcat
\end{verbatim}

Nun müssen alle Shell-Skripte ausführbar gemacht werden:

\begin{verbatim}
sudo chown -R tomcat:tomcat /opt/tomcat/
sudo chmod -R u+x /opt/tomcat/bin
\end{verbatim}

\subsubsection*{Konfiguration und Installation des OCSP-Moduls}

Im Ordner \texttt{xipki/xipki-ocsp/} befindet sich ein Unterordner \texttt{tomcat}, welcher die benötigte OCSP-Konfiguration, Zertifikate und CRLs enthält.

Im Verzeichnis \texttt{./tomcat/xipki} befindet sich bereits ein Ordner für CRLs. Zusätzlich muss ein Ordner für die später benötigten Signer-Zertifikate erstellt werden:

\begin{verbatim}
mkdir keycerts
\end{verbatim}

Die zentrale Konfigurationsdatei befindet sich unter
\texttt{./tomcat/xipki/etc/ocsp/ocsp-responder.json}.
Eine funktionierende Beispielkonfiguration:

\begin{verbatim}
{
    "master": true,
    "unknownIssuerBehaviour": "malformedRequest",

    "datasources": [
        {
            "name": "datasource1",
            "conf": {
                "file": "etc/ocsp/database/ocsp-crl-db.properties"
            }
        }
    ],

    "requestOptions": [
        {
            "name": "request1",
            "hashAlgorithms": ["SHA256"],
            "maxRequestListCount": 10,
            "maxRequestSize": 4096,
            "nonce": { "occurrence": "optional" },
            "supportsHttpGet": true,
            "validateSignature": false,
            "versions": ["v1"]
        }
    ],

    "responders": [
        {
            "name": "responder1",
            "mode": "RFC2560",
            "request": "request1",
            "response": "response1",
            "servletPaths": ["/"],
            "signer": "signer1",
            "stores": ["store1"]
        }
    ],

    "responseOptions": [
        {
            "name": "response1",
            "embedCertsMode": "SIGNER",
            "includeRevReason": true
        }
    ],

    "signers": [
        {
            "name": "signer1",
            "algorithms": ["SHA256withRSA"],
            "type": "pkcs12",
            "cert": { "file": "keycerts/ocsp.crt" },
            "caCerts": [
                { "file": "keycerts/intermediate-ca.crt" }
            ],
            "key": "password=htl,keystore=file:keycerts/ocsp.pfx"
        }
    ],

    "stores": [
        {
            "name": "store1",
            "ignoreExpiredCert": true,
            "ignoreNotYetValidCert": true,
            "includeArchiveCutoff": false,
            "includeCrlId": false,
            "retentionInterval": -1,
            "minNextUpdatePeriod": "1d",
            "unknownCertBehaviour": "good",
            "updateInterval": "10m",
            "source": {
                "datasource": "datasource1",
                "type": "crl",
                "conf": { "dir": "crls" }
            }
        }
    ]
}
\end{verbatim}

Die MySQL-Datenbankkonfiguration befindet sich in
\texttt{./tomcat/xipki/etc/ocsp/database/ocsp-crl-db.properties}.

\subsubsection*{Zertifikate}

Für den OCSP-Signer muss ein eigenes Zertifikat erstellt werden.
Es muss die Extended Key Usage \texttt{OCSP Signing} enthalten.

\begin{figure}[H]
    \centering
    \includegraphics[width=0.8\linewidth]{zertrespond}
    \caption{All Certificates in Certificate-Database}
    \label{fig:zertrespond}
\end{figure}

Das Signer-Zertifikat wird im PKCS\#12-Format (\texttt{ocsp.pfx}) exportiert.
Signer-Zertifikat und Intermediate-CA müssen im PEM-Format in den Ordner \texttt{keycerts} kopiert werden.

\subsubsection*{Datenbank aufsetzen}

\begin{verbatim}
CREATE DATABASE IF NOT EXISTS ocspcrl;
CREATE USER 'ocsp'@'%' IDENTIFIED BY 'htl';
GRANT ALL PRIVILEGES ON ocspcrl.* TO 'ocsp'@'%';
FLUSH PRIVILEGES;
\end{verbatim}

\subsubsection*{CRLs laden}

\begin{verbatim}
mkdir /opt/tomcat/xipki/crls/myca-crl
cp ./ca.crl /opt/tomcat/xipki/crls/myca-crl/ca.crl
cp ./intermediate-ca.crt /opt/tomcat/xipki/crls/myca-crl/ca.crt
touch /opt/tomcat/xipki/crls/myca-crl/UPDATEME
\end{verbatim}

\subsubsection*{Starten und Testen}

\begin{verbatim}
./opt/tomcat/bin/startup.sh
\end{verbatim}

OpenSSL-Test:

\begin{verbatim}
openssl ocsp \
 -issuer intermediate-ca.crt \
 -cert server.crt \
 -url http://localhost:8080/ocsp/ \
 -text \
 -CAfile root-ca.crt
\end{verbatim}

\subsubsection*{System Service}

\begin{verbatim}
sudo nano /etc/systemd/system/tomcat.service
\end{verbatim}

\begin{verbatim}
[Unit]
Description=Tomcat
After=network.target

[Service]
Type=forking
User=tomcat
Group=tomcat

Environment="JAVA_HOME=<java-install>"
Environment="JAVA_OPTS=-Djava.security.egd=file:///dev/urandom"
Environment="CATALINA_BASE=/opt/tomcat"
Environment="CATALINA_HOME=/opt/tomcat"
Environment="CATALINA_PID=/opt/tomcat/temp/tomcat.pid"

ExecStart=/opt/tomcat/bin/startup.sh
ExecStop=/opt/tomcat/bin/shutdown.sh

RestartSec=10
Restart=always

[Install]
WantedBy=multi-user.target
\end{verbatim}

\begin{verbatim}
sudo systemctl start tomcat
\end{verbatim}

\subsection{Verhalten moderner Browser bei OCSP}
\subsubsection{Mozilla Firefox}

Mozilla Firefox nutzt standardmäßig kein OCSP mehr für das checken von widerrufenen Zertifikaten, weil es den sogenannten OpenSSl heartbeat bug gab.  

\subsubsection*{OneCRL}
Anstelle des klassischen OCSP verwendet Mozilla Firefox eine zentralisierte Liste mit widerrufenen Zertifikaten namens OneCRL. Die Liste wird bei jedem Browserupdate aktualisiert und die neuen, von CAs widerrufenen Zertifikate hinzugefügt. 
Vorteile von OneCRL:

\begin{itemize}
	\item \textbf{Zentralisierte Liste}
	\item \textbf{Performance und Schnelligkeit}
	\item \textbf{Automatisch Aktualisierung durch Update}
\end{itemize}

Für ein besseres Verständnis kann die OneCRL-List per JSON heruntergeladen oder als Web Page unter \texttt{https://crt.sh/mozilla-onecrl} angesehen werden.

\begin{figure}[H]
	\centering
	\includegraphics[width=0.8\linewidth]{onecrl-firefox.png}
	\caption{OneCRL in Mozilla Firefox}
	\label{fig:screenshot20}
\end{figure}

\subsubsection*{OCSP-Stapling}
Dennoch kann ihm Mozilla Firefox Browser OCSP-Stapling, also sogenannte Life Querys erzwungen werden. Falls diese jedoch fehlschlagen wird auf die lokale OneCRLListe
zurückgegriffen. Diese Verhalten wird als \textbf{soft-fail} bezeichnet, da nicht
sofort blockiert wird.
Um diese OCSP Checks zu aktivieren muss zu \texttt{about:config security.OCSP.require} navigiert
und \texttt{true} gesetzt werden. Danach kommt bei nicht vorhandener korrekten OCSP-Antwort ein Fehler/Error zurück und die Verbindung ist nicht sicher. 

\begin{figure}[H]
	\centering
	\includegraphics[width=0.8\linewidth]{ocsp-require-firefox.png}
	\caption{Aktivieren von OCSP require}
	\label{fig:screenshot20}
\end{figure}

\paragraph*{Quellen}

\begin{itemize}
	\item SSL.com Support Team (2021). How Do Browsers Handl Revoked SSL/TLS Certificates? \\
	\url{https://www.ssl.com/blogs/how-do-browsers-handle-revoked-ssl-tls-certificates/}
	
	\item Steve Gibson (2014). \textit{Security Certificate Revocation Awareness
	The case for “OCSP Must-Staple”} \\
	\url{https://www.grc.com/revocation/ocsp-must-staple.htm}
	
	\item wiki.mozilla.org (2024). \textit{CA/History of Revocation Checking} \\
	\url{https://wiki.mozilla.org/CA/History_of_Revocation_Checking}
\end{itemize}
\subsubsection{Google Chrome}
\input{chapters/06-Revokation/OCSP_im_Browser/OCSPChrome.tex}
\subsubsection{Microsoft Edge}


Microsoft Edge basiert auf der Chromium-Plattform und übernimmt daher weitgehend die gleichen Designentscheidungen wie Google Chrome. Dies betrifft insbesondere die Behandlung klassischer Widerrufsmechanismen in X.509-basierten Public-Key-Infrastrukturen. Obwohl in Zertifikaten über das AIA-Feld eindeutige OCSP-Responder-Informationen hinterlegt sind, nutzt Edge diese standardmäßig nicht. Statt direkter Widerrufsabfragen setzt Edge primär auf alternative, zentral gepflegte Widerrufssysteme.

\subsubsection*{Verhalten von Microsoft Edge}

Edge führt im Standardbetrieb keine OCSP-Onlineabfragen durch. Die im Zertifikat angegebenen OCSP-Responder werden ignoriert, selbst wenn sie korrekt erreichbar und performant sind. Microsoft verweist auf dieselben Gründe wie Google für dieses Verhalten:

\begin{itemize}
    \item \textbf{Performance}: Jede OCSP-Abfrage würde eine zusätzliche Netzwerkoperation erzeugen und den Seitenaufbau verzögern.
    \item \textbf{Ausfallsicherheit}: Wenn ein OCSP-Responder nicht verfügbar ist, darf dies die Webseitenladung nicht blockieren.
    \item \textbf{Datenschutz}: OCSP-Abfragen würden einem externen Server mitteilen, welche Webseiten ein Nutzer besucht.
\end{itemize}

Statt OCSP nutzt Edge – wie Chrome – vorrangig lokale, von Google bereitgestellte Listen kritischer Widerrufe (CRLSets). Diese enthalten jedoch nur wenige hochprioritäre Sperrungen und ersetzen keinen vollständigen Widerrufsmechanismus.

\paragraph*{OCSP-Stapling}

Edge unterstützt vollständig OCSP-Stapling. Dabei liefert der Webserver eine signierte OCSP-Antwort direkt im TLS-Handshake mit. Dieser Mechanismus wird vom Browser als vertrauenswürdig akzeptiert und ist:

\begin{itemize}
    \item schnell, da keine zusätzlichen Netzwerkabfragen nötig sind,
    \item datenschutzfreundlich, da keine Nutzerinformationen an OCSP-Server gesendet werden,
    \item unabhängig von der aktuellen Erreichbarkeit des OCSP-Responders.
\end{itemize}

OCSP-Stapling ist damit der einzige OCSP-basierte Widerrufsmechanismus, der im Standardverhalten von Edge ohne Einschränkungen genutzt wird.

\subsubsection*{Richtlinienbasierte Aktivierung von OCSP}

Edge bietet Unternehmen die Möglichkeit, klassische OCSP-Abfragen über Gruppenrichtlinien wieder zu aktivieren. Diese Funktion ist jedoch bewusst standardmäßig deaktiviert.

\paragraph*{Relevante Richtlinien}

\begin{itemize}
    \item \texttt{EnableOnlineRevocationChecks} – aktiviert klassische Widerrufsprüfungen über CRL und OCSP.
    \item \texttt{RequireOnlineRevocationChecksForLocalAnchors} – erzwingt CRL-/OCSP-Abfragen für Zertifikate, die von internen CAs ausgestellt wurden.
\end{itemize}

Erst wenn diese Richtlinien gesetzt sind, führt Edge wieder tatsächliche OCSP-Abfragen an den im Zertifikat hinterlegten Responder durch.

\subsubsection*{Problemstellung: Private CA und Widerruf in Edge}

Für interne PKI-Umgebungen – wie in unserem Laboraufbau – ergeben sich aus dem Standardverhalten mehrere Probleme:

\subparagraph*{Edge nutzt den OCSP-Responder nicht}
Selbst wenn ein korrekt konfigurierter OCSP-Responder im AIA-Feld angegeben ist, stellt Edge ohne Richtlinien keinerlei Abfragen.

\subparagraph*{CRLSets ignorieren interne Zertifikate}
Da CRLSets nur öffentliche Zertifizierungsstellen berücksichtigen, werden Zertifikate aus privaten CAs grundsätzlich nicht betrachtet.

\subparagraph*{Kein Widerrufsnachweis ohne Richtlinien}
Ein widerrufenes Zertifikat wird von Edge weiterhin als gültig akzeptiert, wenn:
\begin{itemize}
    \item keine Onlineprüfungen aktiviert wurden,
    \item die PKI auf interne Komponenten beschränkt ist,
    \item kein OCSP-Stapling verwendet wird.
\end{itemize}

Damit gilt: Ohne eine bewusst gesetzte Unternehmensrichtlinie oder OCSP-Stapling bleibt der Widerruf interner Zertifikate für Edge unsichtbar.

\subsubsection{Apple Safari}
\input{chapters/06-Revokation/OCSP_im_Browser/OCSPSafari.tex}



