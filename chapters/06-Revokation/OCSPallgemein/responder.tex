

Für diesen OCSP-Responder wird ein Modell der öffentlich verfügbaren Software \textbf{XiPKI} verwendet, zu finden unter:
\url{https://github.com/xipki/xipki}

\subsubsection*{Download von XiPKI}

Unter den Releases kann das Setup für die zum Zeitpunkt dieser Anleitung aktuellste Version (6.5.3) heruntergeladen werden.
Hinweis: Für Tomcat wird mindestens Java 11 benötigt.

\url{https://github.com/xipki/xipki/releases/tag/v6.5.3}

Das Archiv enthält eine Datei \texttt{restore.sh}, die mit einem Zielverzeichnis ausgeführt werden muss. In diesem Zielverzeichnis landen anschließend alle Module von XiPKI.

\begin{verbatim}
chmod +x restore.sh
./restore.sh ../xipki/
\end{verbatim}

Nach dem Wechsel in dieses Zielverzeichnis ergibt sich folgende Ordnerstruktur:

\begin{verbatim}
-rw-rw-r-- 1 buchi buchi  24K Jan  1 2024 CHANGELOG.md
-rw-rw-r-- 1 buchi buchi 6.8K Jan  1 2024 INSTALL.md
drwxrwxr-x 2 buchi buchi 4.0K Nov 18 22:31 license
-rwxr-xr-x 1 buchi buchi 3.1K Jan  1 2024 prepare.sh
-rw-rw-r-- 1 buchi buchi 7.5K Jan  1 2024 README.md
drwxrwxr-x 3 buchi buchi 4.0K Nov 18 22:31 setup
drwxrwxr-x 5 buchi buchi 4.0K Nov 18 22:31 xipki-ca
drwxrwxr-x 9 buchi buchi 4.0K Nov 18 22:31 xipki-cli
drwxrwxr-x 5 buchi buchi 4.0K Nov 18 22:31 xipki-gateway
drwxrwxr-x 5 buchi buchi 4.0K Nov 18 22:31 xipki-hsmproxy
drwxrwxr-x 9 buchi buchi 4.0K Nov 18 22:31 xipki-mgmt-cli
drwxrwxr-x 5 buchi buchi 4.0K Nov 18 22:31 xipki-ocsp
\end{verbatim}

Für die Installation wird der Ordner \texttt{xipki-ocsp} benötigt.

\subsubsection*{Installation des OCSP-Responders}

\subsubsection*{Installation von Tomcat}

Eine Installationsanleitung findet sich hier:
\url{https://www.digitalocean.com/community/tutorials/how-to-install-apache-tomcat-10-on-ubuntu-20-04}

Für den OCSP-Server wird Tomcat 10 als Webserver benötigt. Es wird empfohlen, Tomcat unter einem eigenen Benutzer auszuführen:

\begin{verbatim}
sudo useradd -m -d /opt/tomcat -U -s /bin/false tomcat
\end{verbatim}

Anschließend wird die neueste Tomcat-Version heruntergeladen und nach \texttt{/opt/tomcat} verschoben:

\begin{verbatim}
wget https://dlcdn.apache.org/tomcat/tomcat-10/v10.1.49/bin/apache-tomcat-10.1.49.tar.gz
tar -xvzf apache-tomcat-10.1.49.tar.gz

sudo mkdir -p /opt/tomcat
sudo mv apache-tomcat-10.1.49 /opt/tomcat
\end{verbatim}

Nun müssen alle Shell-Skripte ausführbar gemacht werden:

\begin{verbatim}
sudo chown -R tomcat:tomcat /opt/tomcat/
sudo chmod -R u+x /opt/tomcat/bin
\end{verbatim}

\subsubsection*{Konfiguration und Installation des OCSP-Moduls}

Im Ordner \texttt{xipki/xipki-ocsp/} befindet sich ein Unterordner \texttt{tomcat}, welcher die benötigte OCSP-Konfiguration, Zertifikate und CRLs enthält.

Im Verzeichnis \texttt{./tomcat/xipki} befindet sich bereits ein Ordner für CRLs. Zusätzlich muss ein Ordner für die später benötigten Signer-Zertifikate erstellt werden:

\begin{verbatim}
mkdir keycerts
\end{verbatim}

Die zentrale Konfigurationsdatei befindet sich unter
\texttt{./tomcat/xipki/etc/ocsp/ocsp-responder.json}.
Eine funktionierende Beispielkonfiguration:

\begin{verbatim}
{
    "master": true,
    "unknownIssuerBehaviour": "malformedRequest",

    "datasources": [
        {
            "name": "datasource1",
            "conf": {
                "file": "etc/ocsp/database/ocsp-crl-db.properties"
            }
        }
    ],

    "requestOptions": [
        {
            "name": "request1",
            "hashAlgorithms": ["SHA256"],
            "maxRequestListCount": 10,
            "maxRequestSize": 4096,
            "nonce": { "occurrence": "optional" },
            "supportsHttpGet": true,
            "validateSignature": false,
            "versions": ["v1"]
        }
    ],

    "responders": [
        {
            "name": "responder1",
            "mode": "RFC2560",
            "request": "request1",
            "response": "response1",
            "servletPaths": ["/"],
            "signer": "signer1",
            "stores": ["store1"]
        }
    ],

    "responseOptions": [
        {
            "name": "response1",
            "embedCertsMode": "SIGNER",
            "includeRevReason": true
        }
    ],

    "signers": [
        {
            "name": "signer1",
            "algorithms": ["SHA256withRSA"],
            "type": "pkcs12",
            "cert": { "file": "keycerts/ocsp.crt" },
            "caCerts": [
                { "file": "keycerts/intermediate-ca.crt" }
            ],
            "key": "password=htl,keystore=file:keycerts/ocsp.pfx"
        }
    ],

    "stores": [
        {
            "name": "store1",
            "ignoreExpiredCert": true,
            "ignoreNotYetValidCert": true,
            "includeArchiveCutoff": false,
            "includeCrlId": false,
            "retentionInterval": -1,
            "minNextUpdatePeriod": "1d",
            "unknownCertBehaviour": "good",
            "updateInterval": "10m",
            "source": {
                "datasource": "datasource1",
                "type": "crl",
                "conf": { "dir": "crls" }
            }
        }
    ]
}
\end{verbatim}

Die MySQL-Datenbankkonfiguration befindet sich in
\texttt{./tomcat/xipki/etc/ocsp/database/ocsp-crl-db.properties}.

\subsubsection*{Zertifikate}

Für den OCSP-Signer muss ein eigenes Zertifikat erstellt werden.
Es muss die Extended Key Usage \texttt{OCSP Signing} enthalten.

\begin{figure}[H]
    \centering
    \includegraphics[width=0.8\linewidth]{zertrespond}
    \caption{All Certificates in Certificate-Database}
    \label{fig:zertrespond}
\end{figure}

Das Signer-Zertifikat wird im PKCS\#12-Format (\texttt{ocsp.pfx}) exportiert.
Signer-Zertifikat und Intermediate-CA müssen im PEM-Format in den Ordner \texttt{keycerts} kopiert werden.

\subsubsection*{Datenbank aufsetzen}

\begin{verbatim}
CREATE DATABASE IF NOT EXISTS ocspcrl;
CREATE USER 'ocsp'@'%' IDENTIFIED BY 'htl';
GRANT ALL PRIVILEGES ON ocspcrl.* TO 'ocsp'@'%';
FLUSH PRIVILEGES;
\end{verbatim}

\subsubsection*{CRLs laden}

\begin{verbatim}
mkdir /opt/tomcat/xipki/crls/myca-crl
cp ./ca.crl /opt/tomcat/xipki/crls/myca-crl/ca.crl
cp ./intermediate-ca.crt /opt/tomcat/xipki/crls/myca-crl/ca.crt
touch /opt/tomcat/xipki/crls/myca-crl/UPDATEME
\end{verbatim}

\subsubsection*{Starten und Testen}

\begin{verbatim}
./opt/tomcat/bin/startup.sh
\end{verbatim}

OpenSSL-Test:

\begin{verbatim}
openssl ocsp \
 -issuer intermediate-ca.crt \
 -cert server.crt \
 -url http://localhost:8080/ocsp/ \
 -text \
 -CAfile root-ca.crt
\end{verbatim}

\subsubsection*{System Service}

\begin{verbatim}
sudo nano /etc/systemd/system/tomcat.service
\end{verbatim}

\begin{verbatim}
[Unit]
Description=Tomcat
After=network.target

[Service]
Type=forking
User=tomcat
Group=tomcat

Environment="JAVA_HOME=<java-install>"
Environment="JAVA_OPTS=-Djava.security.egd=file:///dev/urandom"
Environment="CATALINA_BASE=/opt/tomcat"
Environment="CATALINA_HOME=/opt/tomcat"
Environment="CATALINA_PID=/opt/tomcat/temp/tomcat.pid"

ExecStart=/opt/tomcat/bin/startup.sh
ExecStop=/opt/tomcat/bin/shutdown.sh

RestartSec=10
Restart=always

[Install]
WantedBy=multi-user.target
\end{verbatim}

\begin{verbatim}
sudo systemctl start tomcat
\end{verbatim}
