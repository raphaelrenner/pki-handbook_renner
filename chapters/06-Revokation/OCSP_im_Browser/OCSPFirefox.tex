
Mozilla Firefox nutzt standardmäßig kein OCSP mehr für das checken von widerrufenen Zertifikaten, weil es den sogenannten OpenSSl heartbeat bug gab.  

\subsubsection*{OneCRL}
Anstelle des klassischen OCSP verwendet Mozilla Firefox eine zentralisierte Liste mit widerrufenen Zertifikaten namens OneCRL. Die Liste wird bei jedem Browserupdate aktualisiert und die neuen, von CAs widerrufenen Zertifikate hinzugefügt. 
Vorteile von OneCRL:

\begin{itemize}
	\item \textbf{Zentralisierte Liste}
	\item \textbf{Performance und Schnelligkeit}
	\item \textbf{Automatisch Aktualisierung durch Update}
\end{itemize}

Für ein besseres Verständnis kann die OneCRL-List per JSON heruntergeladen oder als Web Page unter \texttt{https://crt.sh/mozilla-onecrl} angesehen werden.

\begin{figure}[H]
	\centering
	\includegraphics[width=0.8\linewidth]{onecrl-firefox.png}
	\caption{OneCRL in Mozilla Firefox}
	\label{fig:screenshot20}
\end{figure}

\subsubsection*{OCSP-Stapling}
Dennoch kann ihm Mozilla Firefox Browser OCSP-Stapling, also sogenannte Life Querys erzwungen werden. Falls diese jedoch fehlschlagen wird auf die lokale OneCRLListe
zurückgegriffen. Diese Verhalten wird als \textbf{soft-fail} bezeichnet, da nicht
sofort blockiert wird.
Um diese OCSP Checks zu aktivieren muss zu \texttt{about:config security.OCSP.require} navigiert
und \texttt{true} gesetzt werden. Danach kommt bei nicht vorhandener korrekten OCSP-Antwort ein Fehler/Error zurück und die Verbindung ist nicht sicher. 

\begin{figure}[H]
	\centering
	\includegraphics[width=0.8\linewidth]{ocsp-require-firefox.png}
	\caption{Aktivieren von OCSP require}
	\label{fig:screenshot20}
\end{figure}

\paragraph*{Quellen}

\begin{itemize}
	\item SSL.com Support Team (2021). How Do Browsers Handl Revoked SSL/TLS Certificates? \\
	\url{https://www.ssl.com/blogs/how-do-browsers-handle-revoked-ssl-tls-certificates/}
	
	\item Steve Gibson (2014). \textit{Security Certificate Revocation Awareness
	The case for “OCSP Must-Staple”} \\
	\url{https://www.grc.com/revocation/ocsp-must-staple.htm}
	
	\item wiki.mozilla.org (2024). \textit{CA/History of Revocation Checking} \\
	\url{https://wiki.mozilla.org/CA/History_of_Revocation_Checking}
\end{itemize}