\subsubsection*{Ausgangssituation und Zielsetzung}
In diesem Labor wird eine Certificate Revocation List (CRL) erstellt und über einen Webserver (Nginx) bereitgestellt. Die vorhandene CA-Struktur wurde bereits zuvor in XCA aufgebaut. Ziel ist es, ein Serverzertifikat mit einem CRL Distribution Point (CDP) zu versehen, dieses anschließend zu widerrufen und die zugehörige CRL über Nginx zugreifbar zu machen. Am Ende soll der Browser beim Inspizieren des Zertifikats korrekt auf den veröffentlichten CRL-Pfad verweisen.

\subsubsection*{Anpassung des Serverzertifikats in XCA}

Öffnen der bestehenden CA-Struktur in XCA.

\begin{figure}[H]
    \centering
    \includegraphics[width=0.8\textwidth]{xcacrl1.png}
    \caption*{XCA Übersicht}
\end{figure}

Auswahl des entsprechenden Serverzertifikats, das angepasst werden soll.

Im Reiter \texttt{Extensions} bzw. \texttt{Extensions → CRL Distribution Points} den Distribution Point eintragen:

\begin{verbatim}
http://192.168.1.1/crl/cacrl.pem
\end{verbatim}

\begin{figure}[H]
    \centering
    \includegraphics[width=0.8\textwidth]{xcacrl2.png}
    \caption*{CRL Distribution Point}
\end{figure}

\subsubsection*{Widerruf des Serverzertifikats}

Das soeben angepasste Serverzertifikat wird in XCA ausgewählt.

Über \texttt{Revocation / Revoke} wird das Zertifikat widerrufen.

\begin{figure}[H]
    \centering
    \includegraphics[width=0.8\textwidth]{xcacrl3.png}
    \caption*{Widerruf in XCA}
\end{figure}

Dadurch wird klar, dass dieses Zertifikat nicht mehr gültig ist und künftig in der CRL eingetragen werden muss.

\subsubsection*{Erstellen der CRL des Intermediate-Zertifikats}

Da das widerrufene Serverzertifikat vom Intermediate-Zertifikat signiert wurde, wird die CRL auf Ebene dieser Intermediate-CA erzeugt:

\begin{itemize}
    \item Auswahl der Intermediate-CA in XCA.
    \item Im Bereich \texttt{CRLs} die Option \texttt{New CRL} wählen.
    \item CRL-Parameter wie Gültigkeitsdauer übernehmen oder anpassen.
    \item CRL erzeugen und als Datei exportieren (PEM-Format).
\end{itemize}

Ausgabeformat:

\begin{verbatim}
cacrl.pem
\end{verbatim}

\begin{figure}[H]
    \centering
    \includegraphics[width=0.8\textwidth]{xcacrl4.png}
    \caption*{CRL Erstellung}
\end{figure}
