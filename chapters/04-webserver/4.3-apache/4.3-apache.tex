\section{Apache}
Die Konfiguration auf Apache unterscheidet sich in einigen Punkten von der Konfiguration auf Nginx. Apache verwendet sogenannte \textit{Virtual Hosts}, um mehrere Websites auf einem einzigen Server zu hosten. Jeder Virtual Host kann individuell konfiguriert werden, einschließlich der SSL/TLS-Einstellungen.

\subsection{Installieren von Apache}
Um Apache auf einem Debian-basierten System zu installieren, verwenden Sie den folgenden Befehl:

\begin{tcolorbox}[colback=black!3!white, colframe=black!60!white]
\begin{verbatim}
sudo apt install apache2
\end{verbatim}
\end{tcolorbox}

Im nächsten Schritt sollen die Ports 80 (HTTP) und 443 (HTTPS) in der Firewall freigegeben werden und der Server gestartet werden:
\begin{tcolorbox}[colback=black!3!white, colframe=black!60!white]
\begin{verbatim}
//Öffne Port 80 und 443
sudo ufw allow 'Apache Full'
//Starte Apache
sudo systemctl start apache2
sudo systemctl enable apache2
\end{verbatim}
\end{tcolorbox}

Der Webserver sollte nun unter der IP-Adresse des Servers erreichbar sein. Auf jedenfall sollte der Webserver "Active (Running)" sein. 
\begin{tcolorbox}[colback=black!3!white, colframe=black!60!white]
\begin{verbatim}
sudo systemctl status apache2
\end{verbatim}
\end{tcolorbox}

\subsection{Einbinden der Zertifikate in Apache}
Dieser Part unterscheidet sich nicht von Nginx. Auch hier müssen die Zertifikate aus XCA exportiert und auf dem Server gespeichert werden. Die Schritte dazu sind in Abschnitt \ref{sec:export-der-zertifikate-und-schluessel} beschrieben.

\subsection{Apache Konfiguration}
Für die Einbettung in die Config von Apache kann nun die /etc/apache2/sites-available/000-default.conf bearbeitet werden oder eine neue Datei erstellt werden.
\begin{tcolorbox}[colback=black!3!white, colframe=black!60!white]
\begin{verbatim}
sudo nano /etc/apache2/sites-available/000-default.conf
    === Inhalt der Datei ===
    <VirtualHost *:80>
        ServerName mytestsite.test
        DocumentRoot /var/www/<Ordner in dem die Seite liegt>
        # Optional: Redirect auf HTTPS
        Redirect permanent / https://mytestsite.test/
    </VirtualHost>


    <VirtualHost *:443>
        ServerName mytestsite.test
        DocumentRoot /var/www/<Ordner in dem die Seite liegt>


        SSLEngine on
        SSLCertificateFile /etc/ssl/certs/fullchain.pem
        SSLCertificateKeyFile /etc/ssl/private/server.pem


        # Sicherheits-/Protokoll-Einstellungen (modern)
        SSLProtocol -all +TLSv1.2 +TLSv1.3
        SSLCipherSuite HIGH:!aNULL:!MD5
        SSLHonorCipherOrder on


        <Directory /var/www/<Ordner in dem die Seite liegt>>
            Options -Indexes +FollowSymLinks
            AllowOverride All
            Require all granted
        </Directory>


    ErrorLog ${APACHE_LOG_DIR}/mytestsite_error.log
    CustomLog ${APACHE_LOG_DIR}/mytestsite_access.log combined
 </VirtualHost>
\end{verbatim}
\end{tcolorbox}

\pagebreak
Module aktivieren und Site aktivieren:
\begin{tcolorbox}[colback=black!3!white, colframe=black!60!white]
 \begin{verbatim}
 # SSL-Modul aktivieren
 sudo a2enmod ssl
 # Header-Modul falls HSTS verwendet werden soll
sudo a2enmod headers


# Site aktivieren
sudo a2ensite mytestsite.conf


# Falls die Standard-HTTP-Site nicht gebraucht wird:
# sudo a2dissite 000-default.conf
\end{verbatim}
\end{tcolorbox}

Konfiguration testen und Apache neustarten:
\begin{tcolorbox}[colback=black!3!white, colframe=black!60!white]
\begin{verbatim}
# Konfiguration testen
sudo apache2ctl configtest
# Apache neustarten
sudo systemctl restart apache2
\end{verbatim}
\end{tcolorbox}