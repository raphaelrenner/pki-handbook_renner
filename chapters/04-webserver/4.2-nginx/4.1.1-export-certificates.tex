\subsection{Export der Zertifikate und Schlüssel}
\label{sec:export-der-zertifikate-und-schluessel}
Aus der XCA-Datenbank werden folgende Dateien exportiert:

\begin{enumerate}[label=\alph*)]
    \item \textbf{Server-Zertifikat} (\texttt{server.crt})
    \item \textbf{Server Private Key} (\texttt{server.key})
    \item \textbf{Intermediate-Zertifikat} (\texttt{HaMaIntermediateCA.crt})
\end{enumerate}

Beim Export ist darauf zu achten, dass:
\begin{itemize}
    \item das Zertifikat im Format \texttt{PEM (*.crt)} gespeichert wird,
    \item der Private Key im unverschlüsselten Format \texttt{PEM (*.key)} exportiert wird,
    \item und die Dateinamen klar und konsistent benannt sind.
\end{itemize}

Zertifikate exportiert man in XCA folgendermaßen:
\begin{quote}
    \texttt{Click on Certificate → Export}
\end{quote}

\begin{figure}[H]
    \centering
    \includegraphics[width=0.8\linewidth]{17-Export-Server-Certificate.png}
    \caption{Export Server Certificate}
    \label{fig:screenshot17}
\end{figure}

\begin{figure}[H]
    \centering
    \includegraphics[width=0.8\linewidth]{18-Export-Server-Private-Key.png}
    \caption{Export Server Private Key}
    \label{fig:screenshot18}
\end{figure}

\begin{figure}[H]
    \centering
    \includegraphics[width=0.8\linewidth]{19-Export-Intermediate-Certificate.png}
    \caption{Export Intermediate Certificate}
    \label{fig:screenshot19}
\end{figure}