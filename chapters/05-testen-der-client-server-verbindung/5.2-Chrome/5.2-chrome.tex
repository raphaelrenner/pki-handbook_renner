\section{Chrome}

\subsection{Testumgebung}

Als Client-System wird ein Windows~11-Rechner verwendet, auf dem die Verbindung zum Webserver hergestellt wird. 
Die Server-VM basiert auf Ubuntu mit einem installierten Nginx-Webserver, der die zuvor erstellten Zertifikate nutzt.

\begin{itemize}
	\item \textbf{Server:} Ubuntu (Nginx, IP: 192.168.3.210)
	\item \textbf{Client:} Windows 11 (Chrome)
	\item \textbf{Protokoll:} HTTPS (Port 443)
	\item \textbf{Zertifikate:}
	\begin{itemize}
		\item Root CA: \texttt{root\_CA.crt}
		\item Intermediate CA: \texttt{intermediate\_CA.crt}
		\item Server-Zertifikat: \texttt{server.crt}
	\end{itemize}
\end{itemize}

\subsection{Import der Zertifikate}

Damit der Client dem Server-Zertifikat vertraut, muss die Root-Zertifizierungsstelle (Root~CA) im Browser Chrome importiert werden.

\begin{enumerate}
	\item \textit{Einstellungen → Datenschutz und Sicherheit → Sicherheit → Erweitert → Zertifikate verwalten → Von dir installiert → Vertrauenswürdige Zertifikate importieren}
	\item Root~CA und Intermediate~CA auswählen und importieren.
\end{enumerate}

\begin{figure}[H]
	\centering
	\includegraphics[width=0.8\linewidth]{import-into-Chrome.png}
	\caption{Import Root CA into Chrome}
	\label{fig:import-into-Chrome}
\end{figure}

\subsection{Verbindungstest}

Nach erfolgreichem Import muss der Browser neu gestartet werden. Anschließend kann folgende URL aufgerufen werden:

\begin{center}
	\texttt{https://192.168.3.210}
\end{center}
\begin{figure}[H]
	\centering
	\includegraphics[width=0.8\linewidth]{successful-connection-Chrome.png}
	\caption{Successful Connection in Chrome}
	\label{fig:successful-connection-Chrome}
\end{figure}
Die Verbindung sollte nun als sicher angezeigt werden.
