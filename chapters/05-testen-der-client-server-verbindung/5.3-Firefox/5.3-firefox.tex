\section{Firefox}
\subsection{Testumgebung}
Zur Validierung der Zertifikatsgültigkeit wird ein \textbf{Windows 11}-Rechner eingesetzt. Auf Server dient eine Ubuntu-VM, auf welcher ein Nxing-Webserver, mit den importierten Zertifikaten installiert ist. 

\begin{itemize}
	\item \textbf{Server:} Ubuntu (Nginx, IP: 192.168.50.128)
	\item \textbf{Client:} Windows 11 (Mozilla Firefox)
	\item \textbf{Protokoll:} HTTPS (Port 443)
	\item \textbf{Zertifikate:}
	\begin{itemize}
		\item Root CA: \texttt{Ren\_Root\_CA.crt}
		\item Intermediate CA: \texttt{Ren\_Intermediate\_CA.crt}
		\item Server-Zertifikat: \texttt{server.crt}
	\end{itemize}
\end{itemize}

\subsection{Import der Root-CA und des Intermediate-Zertifikats}
Damit der Client eine sichere Verbindung zum Server aufbauen und dem Server-Zertifikat vertraut, muss das Root-Zertifikat in den Firefox Browser importiert werden. 

\begin{enumerate}
	\item Öffnen der Settings im Firefox. 
	\item Unter dem Punkt Privacy \& Security (Datenschutz \& Sicherheit) auf View Certificates (Zertifikate anzeigen) klicken.
	\item Unter dem Reiter Authorities (Zertifizierungsstellen) kann nun mit Klick auf Import die Datei \texttt{Ren\_Root\_CA.crt} importiert werden.
\end{enumerate}

\begin{figure}[H]
	\centering
	\includegraphics[width=0.8\linewidth]{settings-firefox.png}
	\caption{Settings vom Firefox-Browser}
	\label{fig:screenshot20}
\end{figure}

\begin{figure}[H]
	\centering
	\includegraphics[width=0.8\linewidth]{import-cert-into-firefox.png}
	\caption{Importieren des Zertifikats in Firefox}
	\label{fig:screenshot20}
\end{figure}

\begin{figure}[H]
	\centering
	\includegraphics[width=0.8\linewidth]{imported-cert-firefox.png}
	\caption{Importiertes Zertifikat in Firefox}
	\label{fig:screenshot20}
\end{figure}

\subsection{Verbindungstest}
Nach erfolgreichem Import kann folgende URL aufgerufen werden:

\begin{quote}
	\texttt{https://192.168.50.128}
\end{quote}

\begin{figure}[H]
	\centering
	\includegraphics[width=0.8\linewidth]{imported-cert-firefox.png}
	\caption{Importiertes Zertifikat in Firefox}
	\label{fig:screenshot20}
\end{figure}

\begin{figure}[H]
	\centering
	\includegraphics[width=0.8\linewidth]{test-server-connection-firefox.png}
	\caption{Sichere Verbindung zum Server}
	\label{fig:screenshot20}
\end{figure}